\documentclass[11pt]{article}
\usepackage[margin=0.8in]{geometry}
\usepackage{bm}
\usepackage{amsfonts}
\usepackage{amsthm}
\usepackage{amssymb}
\usepackage{amsmath}
\usepackage{xcolor}
\usepackage{tikz}
\usepackage[colorlinks = true]{hyperref}


\title{\textbf{Week 4 Discussion Question Solution}}
\author{UNSW Business School, ACTL3182}
\date{September 2020}

%Comment out for no section colours.
\usepackage{titlesec}
%\titleformat{\section}{\Large\sffamily\bfseries\color{red}}{\thesection}{0.5em}{}[]
%\titleformat{\subsection}{\color{red!75}\large\sffamily\bfseries}{\thesubsection}{0.5em}{}
%\titleformat{\subsubsection}{\color{red!50}\normalsize\sffamily\bfseries}{\thesubsubsection}{0.5em}{}

\begin{document}
	\maketitle
	
	\section*{1. Feedback}
	This week's question was relatively straightfoward so I will be briefer on the feedback.
		\subsection*{Done Well}
			\begin{itemize}
				\item First two parts answered very well, it is clear that you understand the concepts in APT
			\end{itemize}
		\subsection*{Mistakes}
			\begin{itemize}
				\item 	It is insufficient to state that the arbitrage profit in part iii is 0.02. You must state how the arbitrage can be exploited by finding the replicating portfolio and then state which assets should be bought or sold. 
				\item	Incorrect equations for the equilibrium plane of returns eg. (below) writing the return rather than the expected.
				\[	r_{i} = 0.1 + 0.01b_{i,1} + 0.02\lambda_{2}\quad\text{or}\quad	r_{i} = 0.1 + 0.01b_{i,1} + 0.02\lambda_{2} + \xi_{i}			\]
			\end{itemize}
		\subsection*{Concept Tips}
			\begin{itemize}
				\item While this was not penalised, you should have solved the systems by hand since there are only 3 linear equations. Similar questions can appear in an exam, where you may not be able to use software packages.
			\end{itemize}
		%\subsection*{Software Tips}
	\section*{2. Solution}
		\textbf{i. }  Under APT, the expected return of an asset is given by 
		\[	\mu_{i} = \lambda_{0} + \lambda_1 b_{i,1} + \lambda_{2} b_{i,2}.
			\]
		Substituting the values of $\mu_{i}$ and $b_{i,1}, b_{i,2}$ for $i=W,X,Y$, we obtain the linear system:
		\[	\begin{bmatrix}
				1 & 1 & 0.5 \\
				1 & 3 & 0.2 \\
				1 & 3 & -0.5 
				\end{bmatrix}
			\begin{pmatrix}
				\lambda_{0} \\ \lambda_{1} \\ \lambda_{2}
			\end{pmatrix}
			 = 
			 \begin{pmatrix}
			  0.12 \\ 0.134 \\ 0.12
			 \end{pmatrix}
			\]
		which is a standard linear system that can be solved by Gaussian Elimination or calculating the inverse matrix. The solution to this system is 
		\[	\lambda_0 = 0.1,\quad \lambda_{1} = 0.01, \quad\lambda_{2} = 0.02
			\]
		The plane of equilibrium returns is therefore
		\[	\mu_{i} = 0.1 + 0.01b_{i,1} + 0.02 b_{i,2}
			\]
		\textbf{ii. } 
		APT predicts the expected return of $Z$ to be 
		\[	\mu_{Z}^{\text{APT}} = 0.1 + 0.01(2) + 0.02(0) = 0.12,
			\]
		which is not equal to the observed expected return of $\mu_{Z}=0.1$. Hence, $Z$ is mispriced and there is an arbitrage opportunity.\\\\	
		\textbf{iii. } We need to find a replicating portfolio $\bm{w}_{r} = (w_W, w_X, w_Y)$ that has the same exposures as $Z$. Thus, equating the $b$'s of $Z$ and ensuring the weights add to 1, we obtain the three equations
		\[	\begin{cases}
				w_W + w_X + w_Y = 1 \\
				w_W b_{W, 1} + w_X b_{X,1} + w_Y b_{Y,1} = b_{Z, 1} \\ 
				w_W b_{W, 2} + w_X b_{X,2} + w_Y b_{Y,2} = b_{Z, 2}
				\end{cases}	
			\]
		Substituting in the numbers yields the linear system 
		\[	(w_W, w_X, w_Y) \begin{bmatrix}
		1 & 1 & 0.5 \\
		1 & 3 & 0.2 \\
		1 & 3 & -0.5 
		\end{bmatrix} = (1, 2, 0)
			\]
		which has solution $w_W = 0.5$, $w_X = 0$ and $w_Y = 0.5$. Portfolio $\bm{w}_{r}$ has the same exposures as $Z$, so it must have an expected return of $0.12$ under APT, which is higher than the observed expected return of $\mu_Z = 0.1$. Thus, $Z$ is overpriced, and we can buy $\bm{w}_{r}$ and sell $Z$, yielding an expected profit of 
		\[	\mu_{r} - \mu_{Z} = 0.12 - 0.1 = 0.02 > 0.
			\]
		This is an arbitrage profit since the net investment is zero.
	\end{document}